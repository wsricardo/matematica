\documentclass[12pt,a4paper]{article}
\usepackage[utf8]{inputenc}
\usepackage[T1]{fontenc}
\usepackage{amsmath}
\usepackage{amssymb}
\usepackage{makeidx}
\usepackage{graphicx}
\usepackage[left=1.0cm, right=1.0cm, top=1.00cm, bottom=1.50cm]{geometry}
\usepackage[portuguese]{babel}
\title{Anotações do Curso de Métodos Estatísticos e Analise de Dados (Verão 2023 IME/USP) }
\author{Wandeson Ricardo}
\begin{document}
	\maketitle
	Alguns tópicos básicos visto em aula sobre Analise Exloratória de Dados e Metodos Estatísticos.\\
	
	\textbf{Livro texto}
	
	Estatística Básica, Morettin e Bussab. \\
	
	\section{Estatística}
	
	\textbf{Estatística Descritiva} \\


	Estatistica consiste hoje de uma metodologia científica para obtenção, organização e analise de dados.
	 
	 \begin{figure}[h]
	 	\centering
	 	\includegraphics[width=0.6\linewidth]{mestat1}
	 	\caption{}
	 	\label{fig:mestat1}
	 \end{figure}

	\textit{Estatística Descritiva} é parte da estatistica que está interessada na redução, análise e interpretação dos dados sob consideração.\\
	
	\textit{Probabilidade} A Teoria das Probabilidades auxilia na modelagem de fenômenos aleatórios, ou seja, aqueles onde há incerteza.\\
	
	\textit{Inferência Estatística} Conjunto de técnicas que a partir de dados amostrais permite extrair conclusões.\\
	
	\textbf{Amostragem}\\
	
	Associada a coleta de dados desenvolve técnicas para obtenção de um subconjunto de uma população, dita amostras.\\
		 
	\textbf{Técnicas de Amostragem }
	
		\begin{itemize}
			\item Amostragem Aleatória
			\item Amostragem por Conveniência
		\end{itemize}
	\vskip 10pt
 	
 	
 	\section{Análise Exploratória de Dados}
 	
 	A análise exploratória de dados pode ser feita através de \textit{medidas-resumo} ou \textit{técnicas gráficas}.\\
 	
	\textbf{Tipos de Variaveis de Dados} \\
	
		Tipos de variaveis
		
		\begin{itemize}
			\item Qualitativa
				\subitem - Nominal
				\subitem - Ordinal
			\item Quantitativas
				\subitem - Discreta
				\subitem - Contínua
		\end{itemize}
	
	\vskip 10pt
	
	
	\section{Medidas Resumo}
	
	Tipos de medidas. \\
	
	\textbf{Medidas de Posição}\\
	Moda, Mediana, Média aritmética, frequência relativa de uma observação.\\
	
	\textbf{Medidas de Dispersão} \\
	Desvios, Desvio Médio, Desvio Padrão, Variância.
	
	\
		
		
	
\end{document}