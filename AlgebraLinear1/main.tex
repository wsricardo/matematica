\documentclass[12pt]{beamer}
\usetheme{slidemath}

\usepackage[english]{babel}
\usepackage[utf8]{inputenc}
\usepackage{amsmath}
\usepackage{amsthm}
\usepackage{amsfonts}
\usepackage{amssymb}
\usepackage{calrsfs}
\usepackage{calligra}
\usepackage{calrsfs}
\usepackage[mathscr]{euscript}
%use graphics
\usepackage{graphicx}
\usepackage{tikz, tkz-base, tkz-fct}
\usepackage{wrapfig}

\usepackage{subcaption}
\usepackage{pgfplots}
\usepackage{adjustbox}

\setbeamersize{text margin left=4mm,text margin right=4mm}
\setbeamertemplate{navigation symbols}{}

\author{Wandeson Ricardo}
\title{Linear Algebra - Bases}

\begin{document}
	\maketitle
	\begin{frame}{Initial Vector Concepts}
		
	\end{frame}

	\begin{frame}{Vector Space}
		
	\end{frame}

	\begin{frame}{Vector Space Example}
		conteúdo...
	\end{frame}

	\begin{frame}{Vector Subspace}

	\end{frame}

	\begin{frame}{Linear Combinations}
		conteúdo...
	\end{frame}

	\begin{frame}{Linear Indepently}
		conteúdo...
	\end{frame}

	\begin{frame}{Matrices and Linear Combinations}
		conteúdo...
	\end{frame}

	\begin{frame}{Linear Systems and Combinations}
		conteúdo...
	\end{frame}

	\begin{frame}{Linear Transformations}
		conteúdo...
	\end{frame}

	
\end{document}