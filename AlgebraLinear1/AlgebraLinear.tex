\documentclass[12pt,a4paper]{article}
\usepackage[utf8]{inputenc}
\usepackage[T1]{fontenc}
\usepackage{amsmath}
\usepackage{amssymb}
\usepackage{amsthm}
\usepackage{makeidx}
\usepackage{graphicx}
\usepackage[left=1.00cm, right=1.00cm, top=1.00cm, bottom=1.00cm]{geometry}
\usepackage[portuguese]{babel}


\newtheorem{definicao}{Definição}
\title{Anotações de Álgebra Linear}
\author{Wandeson Ricardo}
\begin{document}
\maketitle

\section{Espaços Vetoriais}

\begin{definicao}
	Seja $\mathbb{V}$ um conjunto e $+: \mathbb{V}\times\mathbb{V} \rightarrow \mathbb{V}$ e $\cdot : \mathbb{R} \times \mathbb{V} \rightarrow \mathbb{V}$. Dizemos que $(\mathbb{V}, +, \cdot )$ é um \textbf{\textit{espaço vetorial}}, dados $\forall u,v,w \in \mathbb{V}$ e $\beta, \alpha \in \mathbb{R}$, se satisfaz as seguintes propriedades:\\
	
	\item \underline{Comutatividade} $u + v = v + u$\\
	\item  \underline{Associatividade} $( u + v ) + w = u + (v + w)$\\
										$\alpha \cdot (\beta \cdot v)  = (\alpha \cdot \beta)\cdot  v$
	\item \underline{}
\end{definicao}
	
\end{document}