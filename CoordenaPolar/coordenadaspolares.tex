\documentclass[a4paper, 16pt]{paper}
\usepackage[brazil]{babel}
\usepackage[utf8]{inputenc}
\usepackage{amsmath}
\usepackage{amsthm}
\usepackage{amsfonts}
\usepackage{amssymb}
%use graphics
\usepackage{graphicx}
\usepackage{tikz, tkz-base, tkz-fct}

%\title{Coordenadas Polares}
%\author{Wandeson Ricardo}

\begin{document}
%\maketitle
\section{Sistema de Coordenadas Polar}

% Gráficos Coordenadas Polares

\begin{tikzpicture}[scale=1.2]

	\tkzInit[ xmin=-2, xmax=10,xstep=2,
				ymin=-2, ymax=10,ystep=4]
	
	%\tkzDrawX[label=$x$,noticks] 
	%\tkzDrawY[label=$y$,noticks] 
	\draw [->] (0,0) node[align=left, below]{O} -- (2,2) node[align=left, above]{P};
	\draw [->] (0,0) node[align=left,below]{O} -- (6,0) node[align=left, above]{X};
	\draw [->] (0.9,0) arc (0:45:9mm);
	\draw(1.2,0.2) node[align=right, above]{$\theta$};
	\draw(0.6,1.2) node[align=left, below]{$r$};
				

\end{tikzpicture}

Seja $r \in \mathbb{R}$ e $\theta$ um ângulo em radianos temos o par $(r, \theta)$ que descreve o ponto $P$ no sistema de coordenadas polar onde $r \geq 0$ e $0 \leq \theta \leq 2\pi$ e o ponto $O$ é o polo do sistema. Por definição o sentido anti-horário é positivo, ou seja, $\theta \geq 0$.

% Gráficos Coordenadas Polares

\begin{tikzpicture}[scale=1.2]

	\tkzInit[ xmin=-2, xmax=10,xstep=2,
				ymin=-2, ymax=10,ystep=4]
	
	%\tkzDrawX[label=$x$,noticks] 
	%\tkzDrawY[label=$y$,noticks] 
	\draw [->] (0,0) node[align=left, below]{O} -- (1.98,1.98) node[align=left, above]{P}; % Point P = (2,2).
	\draw [->] (0,0) node[align=left,below]{O} -- (6,0) node[align=left, above]{X};
	\draw [->] (0,0) node[align=left,below]{} -- (0,4) node[align=left, above]{};
	\draw [->] (0,0) -- (2,0); % Vetor OA
	\draw [->] (0,0) -- (0, 2); % Vetor OB
	\draw [dashed] (2,0) -- (2,2);
	\draw [dashed] (0,2) -- (2,2);
	\filldraw (2,2) circle (1pt);
	\draw [->] (0.9,0) arc (0:45:9mm);
	\draw[dashed] (2.82843,0) arc (0:90:2.82843);
	\draw(-0.3,3.9) node(y){$Y$}; % OB
	\draw(2,-0.2) node(x){$A$}; % OA
	\draw(-0.2, 2) node(y){$B$};
	\draw(1.2,0.2) node[align=right, above]{$\theta$};
	\draw(0.6,1.2) node[align=left, below]{$r$};
	
	\draw (1.2,-0.2) node(x){$r cos(\theta)$}; % Vetor OA
	\draw (-0.5, 1.3) node(y){$r sen(\theta)$}; % Vetor OB
	
	
				

\end{tikzpicture}

No gráfico acima observamos que o ponto $P$ tem coordenadas cartesianas $(x,y) = (r cos(\theta), r sen(\theta))$. 
Do Teorema de Pitágoras no triângulo $OAP$ temos que $r^2 = x^2 + y^2$ que é a distância da origem $O$ ao ponto $P$, ou seja, o comprimento do segmento de reta $\overline{OP}$ e também a distância de $O$ até $P$ que é $r$. 

$\newline$

Temos assim as seguintes relações abaixo,


\begin{align*}
x = r cos(\theta) \\
y = r sen(\theta) \\
r^2 = x^2 + y^2 \\
tan(\theta) = \frac{y}{x}
\end{align*}


Tendo $(x,y)$ podemos obter $\theta$ usando $ \theta = arctan(\frac{y}{x}) $. Daí decorre que as coordenadas para o ponto $P=(x,y)$  em coordenadas polar é dada por
 $$(r, \theta ) = (\sqrt{(x^2+y^2)},\; arctan(\frac{y}{x}))$$,
 
onde $\theta$ é dado em radianos.

\end{document}